\documentclass{llncs}
\usepackage[utf8]{inputenc}


\usepackage{biblatex} % Add biblatex package
\addbibresource{references.bib} % Specify the bibliography file

\input{macros}


\renewcommand{\thesection}{Classes \arabic{section}:}

\renewcommand{\thesubsection}{Classes \arabic{section}.\arabic{subsection}}

\title{Cryptography II\\ 2024/25}
\subtitle{Classes}
\author{Pawel Kedzior and Marcin Mielniczuk}


\institute{University of Warsaw}

\begin{document}

\maketitle

\section{Elliptic Curve Cryptography and Bilinear Pairings (part 1)}

\DATE{Feb 24/26}

\begin{itemize}
    \item Tripartite Diffie-Hellman key exchange \cite[Section 3.2]{tdh}.
    \item Proof of the BLS signature scheme \cite[Theorem 15.1]{Cryptobook}.
\end{itemize}


\section{Elliptic Curve Cryptography and Bilinear Pairings (part 2)}

\DATE{Mar 03/05}

\begin{itemize}
    \item Searching on PKE-encrypted data from IBE \cite[Section 15.6.4.3]{Cryptobook}.
    \item Proof of the Boneh-Franklin IBE scheme \cite[Theorem 15.6]{Cryptobook}.
\end{itemize}


\section{Zero-Knowledge and Interactive Proofs (part 1)}
\DATE{Mar 10}
\textbf{Introduction to ZK}
\begin{itemize}
	\item commitment definition with properties
		\begin{enumerate}
			\item biding
			\item hiding
		\end{enumerate}
	\item properties of proof system:
		\begin{enumerate}
			\item completeness
			\item soundness
		\end{enumerate}
	\item zk proof system for Hamiltonian Cycles (no simulator)
	\item zk proof system for  3-colouring (hint about simulator)
	\item intuition on simulator, both from the point of view of ZK and real / ideal world paradigm
\end{itemize}

\section{Zero-Knowledge and Interactive Proofs (part 2)}
\DATE{Mar 17}
\textbf{Simulators and Schwartz-Zippel lemma}
\begin{itemize}
	\item Simulator for 3-colouring ($n|E|$ rounds and constant round - HOMEWORK ZK: for parallelized protocol)
	\item Schwartz-Zippel lemma with proof (will be reminded during lecture)
\end{itemize}

\section{Zero-Knowledge and Interactive Proofs (part 3)}
\DATE{Mar 24}
\textbf{Polynomial commitments}
\begin{itemize}
	\item Introduction to polynomial commitments
	\item Merkle commitment
	\item KZG commitment with proofs of correctness and soundness
\end{itemize}


\section{Zero-Knowledge and Interactive Proofs (part 4)}
\DATE{Mar 31}
\textbf{Polynomial commitments, part 2}
\begin{itemize}
	\item BLS homework
	\item KZG soundness (continuation from previous classes)
	\item KZG batch opening for single polynomial (HOMEWORK: multiple polynomials)
\end{itemize}

\section{Zero-Knowledge and Interactive Proofs (part 5)}
\DATE{Apr 07}
\begin{itemize}
	\item Powers-of-Tau Ceremony
	\item Prescribed permutation check lemma
	\item zk-KZG (will be continued on the next classes)
\end{itemize}

\section{Secure Two-Party and Multiparty Computations (part 1)}
\DATE{Apr 14}
\begin{itemize}
	\item Various properties of MPC (correctness, privacy, independence of inputs, GOD, fairness) \cite{LindellMPC}
	\item Definition of semi-honest MPC (started)
	\item Rabin's OT is equivalent to $(2,1)$-OT, MPC implies OT
	\item MPC implies ZK
\end{itemize}

\section{Secure Two-Party and Multiparty Computations (part 2)}
\DATE{Apr 28}
\begin{itemize}
	\item Finished the definition of semi-honest and malicious MPC
	\item GQ protocol for proving knowledge of a square root in an RSA group \cite[Section 19.5.5]{Cryptobook}
	\item Commitments from $(2,1)$-OT
	\item MPC for sum from secret-sharing
	\item BGW protocol for MPC (sketched the degree reduction procedure)
\end{itemize}


\section{Secure Two-Party and Multiparty Computations (part 3)}
\DATE{May 05}

\section{Lattices and Post-Quantum Cryptography (part 1)}
\DATE{May 12}

\section{Lattices and Post-Quantum Cryptography (part 2)}
\DATE{May 19}

\section{Lattices and Post-Quantum Cryptography (part 3)}
\DATE{May 26}

\section{Goldreich-Levin Theorem}
\DATE{Jun 02}

\section{Randomness Extraction and Leakage-Resilience}
\DATE{Jun 09}


\printbibliography % Add bibliography section


\end{document}